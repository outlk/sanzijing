人~之~初~性~本~善~性~相~近~习~相~远
苟~不~教~性~乃~迁~教~之~道~贵~以~专
昔~孟~母~择~邻~处~子~不~学~断~机~杼
窦~燕~山~有~义~方~教~五~子~名~俱~扬
养~不~教~父~之~过~教~不~严~师~之~惰
子~不~学~非~所~宜~幼~不~学~老~何~为
玉~不~琢~不~成~器~人~不~学~不~知~义
为~人~子~方~少~时~亲~师~友~习~礼~仪
香~九~龄~能~温~席~孝~于~亲~所~当~执
融~四~岁~能~让~梨~弟~于~长~宜~先~知
首~孝~悌~次~见~闻~知~某~数~识~某~文
一~而~十~十~而~百~百~而~千~千~而~万
三~才~者~天~地~人~三~光~者~日~月~星
三~纲~者~君~臣~义~父~子~亲~夫~妇~顺
曰~春~夏~曰~秋~冬~此~四~时~运~不~穷
曰~南~北~曰~西~东~此~四~方~应~乎~中
曰~水~火~木~金~土~此~五~行~本~乎~数
十~干~者~甲~至~癸~十~二~支~子~至~亥
曰~黄~道~日~所~躔~曰~赤~道~当~中~权
赤~道~下~温~暖~极~我~中~华~在~东~北
寒~燠~均~霜~露~改~右~高~原~左~大~海
曰~江~河~曰~淮~济~此~四~渎~水~之~纪
曰~岱~华~嵩~恒~衡~此~五~岳~山~之~名
古~九~州~今~改~制~称~行~省~三~十~五
曰~士~农~曰~工~商~此~四~民~国~之~良
曰~仁~义~礼~智~信~此~五~常~不~容~紊
地~所~生~有~草~木~此~植~物~遍~水~陆
有~虫~鱼~有~鸟~兽~此~动~物~能~飞~走
稻~梁~菽~麦~黍~稷~此~六~谷~人~所~食
马~牛~羊~鸡~犬~豕~此~六~畜~人~所~饲
曰~喜~怒~曰~哀~惧~爱~恶~欲~七~情~俱
青~赤~黄~及~黑~白~此~五~色~目~所~识
酸~苦~甘~及~辛~咸~此~五~味~口~所~含
膻~焦~香~及~腥~朽~此~五~臭~鼻~所~嗅
匏~土~革~木~石~金~丝~与~竹~乃~八~音
曰~平~上~曰~去~入~此~四~声~宜~调~协
高~曾~祖~父~而~身~身~而~子~子~而~孙
自~子~孙~至~玄~曾~乃~九~族~人~之~伦
父~子~恩~夫~妇~从~兄~则~友~弟~则~恭
长~幼~序~友~与~朋~君~则~敬~臣~则~忠
此~十~义~人~所~同~当~师~叙~勿~违~背
斩~齐~衰~大~小~功~至~缌~麻~五~服~终
礼~乐~射~御~书~数~古~六~艺~今~不~具
惟~书~学~人~共~遵~既~识~字~讲~说~文
有~古~文~大~小~篆~隶~草~继~不~可~乱
若~广~学~惧~其~繁~但~略~说~能~知~原
凡~训~蒙~须~讲~究~详~训~诂~明~句~读
为~学~者~必~有~初~小~学~终~至~四~书
论~语~者~二~十~篇~群~弟~子~记~善~言
孟~子~者~七~篇~止~讲~道~德~说~仁~义
作~中~庸~乃~孔~伋~中~不~偏~庸~不~易
作~大~学~乃~曾~子~自~修~齐~至~平~治
中~书~熟~孝~经~通~如~六~经~始~可~读
诗~书~易~礼~春~秋~号~六~经~当~讲~求
有~连~山~有~归~藏~有~周~易~三~易~详
有~典~谟~有~训~诰~有~誓~命~书~之~奥
我~周~公~作~周~礼~著~六~官~存~治~体
大~小~戴~注~礼~记~述~圣~言~礼~乐~备
有~国~风~有~雅~颂~号~四~诗~当~讽~咏
诗~既~亡~春~秋~作~寓~褒~贬~别~善~恶
三~传~者~有~公~羊~有~左~氏~有~谷~梁
尔~雅~者~善~辨~言~求~经~训~此~莫~先
古~圣~著~先~贤~传~注~疏~备~十~三~经
左~传~外~有~国~语~合~群~经~数~十~五
经~既~明~方~读~子~撮~其~要~记~其~事
五~子~者~有~荀~扬~文~中~子~及~老~庄
经~子~通~读~诸~史~考~世~系~知~终~始
自~羲~农~至~黄~帝~号~三~皇~在~上~世
唐~有~虞~号~二~帝~相~揖~逊~称~盛~世
夏~有~禹~商~有~汤~周~文~武~称~三~王
夏~传~子~家~天~下~四~百~载~迁~夏~社
汤~伐~夏~国~号~商~六~百~载~至~纣~亡
周~武~王~始~诛~纣~八~百~载~最~长~久
周~共~和~始~纪~年~历~宣~幽~遂~东~迁
周~道~衰~王~纲~坠~逞~干~戈~尚~游~说
始~春~秋~终~战~国~五~霸~强~七~雄~出
嬴~秦~氏~始~兼~并~传~二~世~楚~汉~争
高~祖~兴~汉~业~建~至~孝~平~王~莽~篡
光~武~兴~为~东~汉~四~百~年~终~于~献
魏~蜀~吴~争~汉~鼎~号~三~国~迄~两~晋
宋~齐~继~梁~陈~承~为~南~朝~都~金~陵
北~元~魏~分~东~西~宇~文~周~兴~高~齐
迨~至~隋~一~土~宇~不~再~传~失~统~绪
唐~高~祖~起~义~师~除~隋~乱~创~国~基
二~十~传~三~百~载~梁~灭~之~国~乃~改
梁~唐~晋~及~汉~周~称~五~代~皆~有~由
赵~宋~兴~受~周~禅~十~八~传~南~北~混
辽~与~金~皆~称~帝~元~灭~金~绝~宋~世
舆~图~广~超~前~代~九~十~年~国~祚~废
迨~成~祖~迁~燕~京~十~六~世~至~崇~祯
权~阉~肆~寇~如~林~李~闯~出~神~器~焚
清~世~祖~膺~景~命~靖~四~方~克~大~定
由~康~雍~历~乾~嘉~民~安~富~治~绩~夸
道~咸~间~变~乱~起~始~英~法~扰~都~鄙
同~光~后~宣~统~弱~传~九~帝~满~清~殁
革~命~兴~废~帝~制~立~宪~法~建~民~国
古~今~史~全~在~兹~载~治~乱~知~兴~衰
史~虽~繁~读~有~次~史~记~一~汉~书~二
后~汉~三~国~志~四~兼~证~经~参~通~鉴
读~史~者~考~实~录~通~古~今~若~亲~目
口~而~诵~心~而~惟~朝~于~斯~夕~于~斯
昔~仲~尼~师~项~橐~古~圣~贤~尚~勤~学
赵~中~令~读~鲁~论~彼~既~仕~学~且~勤
披~蒲~编~削~竹~简~彼~无~书~且~知~勉
头~悬~梁~锥~刺~股~彼~不~教~自~勤~苦
如~囊~萤~如~映~雪~家~虽~贫~学~不~辍
如~负~薪~如~挂~角~身~虽~劳~犹~苦~卓
苏~老~泉~二~十~七~始~发~愤~读~书~籍
彼~既~老~犹~悔~迟~尔~小~生~宜~早~思
若~梁~灏~八~十~二~对~大~廷~魁~多~士
彼~既~成~众~称~异~尔~小~生~宜~立~志
莹~八~岁~能~咏~诗~泌~七~岁~能~赋~棋
彼~颖~悟~人~称~奇~尔~幼~学~当~效~之
蔡~文~姬~能~辩~琴~谢~道~韫~能~咏~吟
彼~女~子~且~聪~敏~尔~男~子~当~自~警
唐~刘~晏~方~七~岁~举~神~童~作~正~字
彼~虽~幼~身~已~仕~有~为~者~亦~若~是
犬~守~夜~鸡~司~晨~苟~不~学~曷~为~人
蚕~吐~丝~蜂~酿~蜜~人~不~学~不~如~物
幼~习~业~壮~致~身~上~匡~国~下~利~民
扬~名~声~显~父~母~光~于~前~裕~于~后
人~遗~子~金~满~赢~我~教~子~唯~一~经
勤~有~功~戏~无~益~戒~之~哉~宜~勉~力
